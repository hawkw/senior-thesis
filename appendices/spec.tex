% Appendix A: Mnemosyne Grammar
% $Id: app-spec
%
%   *******************************************************************
%   * SEE THE MAIN FILE "main.tex" FOR MORE INFORMATION.       *
%   *******************************************************************

\chapter{The Mnemosyne Programming Language: An Abridged Description}\label{app:spec}

The following is an abbreviated description of the Mnemosyne programming language. It is for illustrative purposes only and is not intended as a complete formal specification.

\section{Program Structure}

All Mnemosyne programs consist of one or more \textit{modules}. A module forms the top level of a Mnemosyne program and represents a namespace within which types and functions may be defined. A module then consists of a series of one or more \textit{definitions} and \textit{expressions}. An expression is a value-level construct: all expressions can be evaluated to some value, either at run-time or at compile-time. Definitions, by contrast, are type-level constructs.

\section{Syntax}

Mnemosyne expressions may be written using either the traditional S-expression syntax or the indentation-sensitive I-expressions syntax.

\subsection{Notation}
Syntax descriptions are written using an extended BNF notation, as follows:
\begin{description}
    \item{\synt{symbol}} indicates a non-terminal symbol
    \item{\lit{symbol}} indicates a terminal symbol
    \item{\synt{symbol}* } indicates zero or more repetitions of \synt{symbol}
    \item{\synt{symbol}+} indicates one or more repetitions of \synt{symbol}.
    \item{$\varepsilon$} indicates the empty string
\end{description}

The following special symbols refer to specific Unicode characters:
\begin{description}
    \item{\synt{lambda}} Greek capital letter Lambda (U+03BB)
    \item{\synt{arrow}} Rightwards arrow (U+2192)
    \item{\synt{double arrow}} Rightwards double arrow (U+21D2)
    % \item{\synt{tab}} Character tabulation (U+0009)
    % \item{\synt{linefeed}} Linefeed (U+000A)
    % \item{\synt{return}} Carriage return (U+000D)
    % \item{\synt{space}} Space (U+0020)
\end{description}

Finally, the symbol \synt{any} refers to any character.

\subsection{Lexical Syntax}
% \newcommand{\into}{$\longrightarrow$}
Note: \Cref{sec:code:parser} contains a source code listing for the Mnemosyne parser, and should be referred to to answer specific questions regarding how the reference Mnemosyne implementation handles specific characters.

\setlength{\grammarindent}{6em}
% \begin{listing}[H]
    \begin{grammar}
        <program> $\to$ <token>+

        <token> $\to$ <lexeme> | <atmosphere>

        <lexeme> $\to$ <identifier> | <operator> | <keyword> | <literal>
                    \alt <sigil> | <delimiter>

        <sigil> $\to$ `@' | `&' | `*' | `\$' | `?'

        <delimiter> $\to$ `(' | `)' | `{' | `}'

        <identifier> $\to$ <initial> <subsequent>*

        <initial> $\to$ <letter> | <special initial>

        <subsequent> $\to$ <letter> | <number> | <special subsequent>

        <letter> $\to$ `a' | `b' | `c' | ... | `z'
                 \alt `A' | `B' | `C' | ... | `Z'

        <number> $\to$ `0' | `1' | ... | `9'

        <special initial> $\to$ `+' | `-' | `*' | `<' | `>' | `='
                          \alt `!' | `:' | `\%' | `^'

        <special subsequent> $\to$ <special initial> | `\'' | `_'

        <keyword> $\to$ `and' | `begin'  | `borrow' | `case' | `cond'
                  \alt `class' | `data' | `define' | `defn' | `def'
                  \alt `delay'| `do' | `else' | `if' | `instance'
                  \alt `impl' | `lambda' | `let' | `let*' | `letrec'
                  \alt `mod' | `or' | `quasiquote' | `quote' | `ref'
                  \alt `set!' | `struct' | `trait' | `type' | `typeclass'
                  \alt `union' |  `unquote' | `unquote-splicing'
                  \alt <lambda> | <arrow> | <fat arrow>
                  \alt <builtin type>
                  \alt \lit{|}  | `->' | `=>` | `,'

        <builtin type> $\to$ `i8' | `i16' | `i32' | `i64' | `int'
                        \alt `u8' | `ui6` | `u32' | `u64' | `uint'
                        \alt `f32' | `f64' | `float' | `double'
                        \alt `bool' | `string'

        <atmosphere> $\to$ <whitespace> | <comment>

        <whitespace> $\to$ ` '
                    \alt tab (U+0009)
                    \alt linefeed (U+000A)
                    \alt carriage return (U+000D)

        <comment> $\to$ `;' <any>* <line ending>
                  \alt \lit{\#|} <any>* \lit{|\#}

    \end{grammar}
    % \caption{Mnemosyne lexical syntax.}
% \end{listing}

\subsection{Program Syntax}

\begin{grammar}
<program> $\to$ <module>+

<module> $\to$ <module-def> <definition>*

<module-def> $\to$ `(' `mod' <identifier> <exports-clause> `)'

<exports-clause> $\to$ `(' `exports' <identifier>+ `)'
                \alt $\varepsilon$


%% skip a bit
<expr> $\to$ <deref expr> | <unwrap expr> | <pointer expr>

<deref expr> $\to$ `(' \lit{\$} <expr> `)'
             \alt \lit{\$} <expr>

<unwrap expr> $\to$ `(' `?' <expr> <expr>`)'
             \alt `(' `?' <expr> `)'
             \alt `?'<expr>

<pointer expr> $\to$ `(' <pointer sigil> <expr> `)'
             \alt <pointer sigil> <expr>

<pointer sigil> $\to$ `&' | `@' | `*'

\end{grammar}
