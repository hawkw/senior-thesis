%
% $Id: ch04_implementation.tex
%
%   *******************************************************************
%   * SEE THE MAIN FILE "AllegThesis.tex" FOR MORE INFORMATION.       *
%   *******************************************************************
%
\chapter{Implementation}\label{ch:implem}
The Mnemosyne compiler, called \texttt{mn}\footnote{Pronounced ``Manganese''.} operates in three primary phases:

\begin{itemize}
\item \textbf{Semantic analysis} converts the program source code to a representation understandable by the compiler and detects syntax errors
\item \textbf{Semantic analysis} attempts to prove statements about the program's execution, such as determining the types of values and reducing expressions to constants, and detects semantic errors
\item \textbf{Code generation} converts the internal representation of the program to LLVM intermediate representation (IR) and then to the desired output binary format.
\end{itemize}

\section{Parsing and Syntactic Analysis}\label{sec:parsing}
The Mnemosyne parser is implemented using technique called \textit{combinator parsing}.

\section{Semantic Analysis}\label{sec:analysis}
\section{Code Generation}\label{sec:codegen}
