\documentclass{beamer}
% \usepackage{listings}
\usepackage{color}
\usepackage{lmodern}
%% -- Referencing -------------------------------------------------------
\usepackage{hyperref}
\usepackage{cleveref}
\usepackage[ backend=bibtex
           , style = numeric
           , sortcites = true
           , url = false
           , doi = false
           , sorting = nty
           , backref = false
           , natbib
           , hyperref
           ]{biblatex}
\bibliography{../bibliography/thesis.bib}
\renewcommand*{\bibfont}{\scriptsize}
\usepackage{lmodern}
\usepackage{minted}
\setminted{ fontfamily = lmtt
          , style = solarizeddark
          , mathescape = true
          }
\usepackage{fancyvrb}

\usecolortheme[dark]{solarized}

\title[Mnemosyne]{Mnemosyne: A~Functional~Language~for~Systems~Programming}
\subtitle{CMPSC600 Senior Thesis Proposal}
\author[Hawk Weisman]{Hawk Weisman}
\institute[Allegheny College]{Department of Computer Science \\ Allegheny College}
\date{November 20\textsuperscript{th}, 2015}
\begin{document}
\maketitle

\begin{frame}
\alert{\huge{Proposal}}\\
\large{To implement and evaluate a prototype compiler for the Mnemosyne programming language.}
\begin{itemize}
    \item Mnemosyne is a functional language for systems programming, with compile-time automatic memory~management.
    \item But what does that mean?
\end{itemize}
\end{frame}

\begin{frame}
\alert{\huge{What is Mnemosyne?}}\\
\large{\alert{A functional language} for systems programming, with compile-time automatic memory~management.} \normalsize
\begin{itemize}
\item<2->
\textbf{Functional programming} models computation as the evaluation of functions~\cite{Wise:2003:FP:1074100.1074416,hughes1989functional}
\begin{itemize}
    \item<2-> \textbf{It focuses on} immutability, purity, and function composition
    \item<3> \textbf{Advantages:} expressiveness~\cite{hughes1989functional,hudak1994haskell}, modularity %(easy to test and parallelize)
    ~\cite{hughes1989functional,hudak1994haskell}, safety
    \end{itemize}
\end{itemize}
\end{frame}

\begin{frame}
\alert{\huge{What is Mnemosyne?}}\\
\large{\alert{A functional language} for systems programming, with compile-time automatic memory~management.} \normalsize
\begin{itemize}
\item
\textbf{Mnemosyne is inspired by:}
    \begin{itemize}
        \item\textbf{Lisp}'s syntax and homoiconicity~\cite{vanderhart2010macros,sicp}.
        \item\textbf{Haskell and ML}'s type system~\cite{jones2003haskell,hudak1992gentle,hudak1992report} and pattern matching~\cite{jones2003haskell,maranget2007warnings,Krishnaswami:2009:FPM:1594834.1480927,hudak1992report}
        \item\textbf{Rust}'s memory management~\cite{whyrust,whyrust}
    \end{itemize}

\end{itemize}
\end{frame}

\begin{frame}
\alert{\huge{What is Mnemosyne?}}\\
\large{A functional language \alert{for systems programming}, with compile-time automatic memory~management.}
\normalsize
\begin{itemize}
\item<1->
    \textbf{Systems programming} is the implementation of software that provide services to other software~\cite{Narten:2003:SP:1074100.1074850,Shapiro:2006:PLC:1215995.1216004}.
    % \begin{itemize}
    %     \item<2->{Operating systems}
    %     \item<2->{Device drivers}
    %     \item<2->{Language runtimes}
    %     \item<2->...
    % \end{itemize}
    \item <2-> High quality systems are necessary for high quality applications.
\item <3-> But there are some significant challenges in this field~\cite{whyrust,Shapiro:2006:PLC:1215995.1216004}
\end{itemize}
\end{frame}

\begin{frame}
\alert{\huge{What is Mnemosyne?}}\\
\large{A functional language for systems programming, \alert{with compile-time automatic memory~management}.}\normalsize
\begin{itemize}
\item Almost all
systems
programming today is done in C~\cite{Shapiro:2006:PLC:1215995.1216004,hawblitzel2004low}

    \item<2-> \textbf{Why?} C manages memory at compile-time
    \begin{itemize}
        \item<3-> Most languages use garbage collection (GC)~\cite{Bartley:2003:GC:1074100.1074419}
        \item<3-> GC is unsuitable for most low-level systems~\cite{hawblitzel2004low,Shapiro:2006:PLC:1215995.1216004,Hertz:2005:QPG:1094811.1094836}
        \item<3-> C manages memory manually (\texttt{malloc()}/\texttt{free()})~\cite{Shapiro:2006:PLC:1215995.1216004,kernighan1988c,Hertz:2005:QPG:1094811.1094836}
    \end{itemize}
\end{itemize}
\end{frame}

\begin{frame}
\alert{\huge{What is Mnemosyne?}}\\
\large{A functional language for systems programming, \alert{with compile-time automatic memory~management}.}\normalsize
\begin{itemize}
    \item \textbf{Manual memory management leads to errors} such as buffer overflows, memory leaks, and null pointer dereferences~\cite{Shapiro:2006:PLC:1215995.1216004,hawblitzel2004low}
    \item \textbf{What if there was another way?}
\end{itemize}
\end{frame}

\begin{frame}
\alert{\huge{What is Mnemosyne?}}\\
\large{A functional language for systems programming, \alert{with compile-time automatic memory~management}.}\normalsize
\begin{itemize}
    \item Mnemosyne manages memory automatically at compile time
    \item \textbf{How?}
    \begin{itemize}
    \item<2-> Stack allocation~\cite{Hanson:1990:ESA:91556.91603,Corry:2006:OSA:1133956.1133978,Matsakis:2014:RL:2663171.2663188}
    \item<2-> Ownership analysis~\cite{Matsakis:2014:RL:2663171.2663188}
    \item<2-> Controlled mutability~\cite{Matsakis:2014:RL:2663171.2663188}
\end{itemize}
\end{itemize}
\end{frame}

\begin{frame}[fragile]
\alert{\huge{Mnemosyne Syntax}}\\
\large{A brief example:}\normalsize
\begin{minted}{clojure}
(def factorial (fn ( -> int int )
    ((factorial 0) 1)
    ((factorial n) ( * n (factorial (- n 1)))
)))
\end{minted}
\end{frame}

\begin{frame}[fragile]
\alert{\huge{Mnemosyne Syntax}}\\
\large{Using syntactic sugar...}\normalsize
\begin{minted}{clojure}
def factorial (fn { int -> int }
    (factorial 0) 1
    (factorial n) {
        n * factorial({n - 1})
    })
\end{minted}
\end{frame}

\begin{frame}
\alert{\huge{Methods}}\\
\large{Manganese, the Mnemosyne compiler, is implemented in Rust}\normalsize
\begin{itemize}
    \item \textbf{Combinator parsing}~\cite{Danielsson:2010:TPC:1932681.1863585,frost2008parser,swierstra2001combinator,hutton1996monadic} using \texttt{combine} and \texttt{combine-language}
    \item \textbf{Analysis} including type checking and lifetime analysis~\cite{Matsakis:2014:RL:2663171.2663188,sobalvarro1988lifetime}
    \item \textbf{Code generation} using \texttt{librustc-llvm}~\cite{Lattner:2004:LCF:977395.977673}
\end{itemize}
\end{frame}

\begin{frame}
\alert{\huge{Methods}}\\
\large{Assessing Mnemosyne's correctness}\normalsize
\begin{itemize}
    \item \textbf{Unit and integration testing} to validate the compiler implementation
    \item \textbf{Demonstration} by implementing example code, including parts of the prelude
    \item \textbf{Benchmarking} compiled Mnemosyne binaries
\end{itemize}
\end{frame}

\begin{frame}
\alert{\huge{Questions?}}\\
\large{For more information:}\normalsize
\begin{itemize}
    \item \textbf{Sample Mnemosyne code} if there's time
    \item \textbf{Complete source code:} \url{https://github.com/hawkw/mnemosyne}
\end{itemize}
\end{frame}

\begin{frame}[t,allowframebreaks]
    \frametitle{\huge{References}}
    \scriptsize
    % \bibliographystyle{plain}
    % \bibliography{../bibliography/thesis.bib}
    \printbibliography
\end{frame}

\end{document}
