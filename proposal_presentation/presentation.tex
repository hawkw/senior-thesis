\documentclass{beamer}
% \usepackage{listings}
\usepackage{color}
\usepackage{lmodern}
%% -- Referencing -------------------------------------------------------
\usepackage{hyperref}
\usepackage{cleveref}
\usepackage[ backend=bibtex
           , style = numeric
           , sortcites = true
           , url = false
        %    , doi = false
           , sorting = nty
           , backref = false
           , natbib
           , hyperref
           ]{biblatex}
\bibliography{../bibliography/thesis.bib}
\renewcommand*{\bibfont}{\scriptsize}

\usecolortheme[dark]{solarized}

\title{\huge Mnemosyne}
\subtitle{A Functional Systems~Programming Language}
\author[Hawk Weisman]{Hawk Weisman}
\institute[Allegheny College]{Department of Computer Science \\ Allegheny College}
\date{\today}
\begin{document}
\maketitle

\begin{frame}
\huge What is Mnemosyne? \normalsize \\
A functional systems programming language with compile-time memory management.
\begin{itemize}
    \item But what does that mean?
\end{itemize}
\end{frame}

\begin{frame}
\huge What is Mnemosyne? \normalsize \\
A \alert<1->{functional} systems programming language with compile-time memory management.
\begin{itemize}
\item<2->
\textbf{Functional programming} models computation as the evaluation of functions~\cite{Wise:2003:FP:1074100.1074416,hughes1989functional}
    \item<2-> Focus on:
    \begin{itemize}
        \item<2-> Immutability
        \item<2-> Purity
        \item<2-> Function composition
    \end{itemize}
    \item<3> Advantages: \begin{itemize}
        \item<3> Expressiveness~\cite{hughes1989functional,hudak1994haskell}
        \item<3> Modular (easy to test and parallelize)~\cite{hughes1989functional,hudak1994haskell}
        \item<3> Safe
    \end{itemize}
\end{itemize}
\end{frame}

\begin{frame}
\huge What is Mnemosyne? \normalsize \\
A functional \alert<1->{systems programming} language with compile-time memory management.
\begin{itemize}
\item<1->
    \textbf{Systems programming} is the implementation of software that provide services to other software~\cite{Narten:2003:SP:1074100.1074850,Shapiro:2006:PLC:1215995.1216004}.
    \begin{itemize}
        \item<2->{Operating systems}
        \item<2->{Device drivers}
        \item<2->{Language runtimes}
        \item<2->...
    \end{itemize}
    \item <3-> High quality systems are necessary for high quality applications.
\item <4-> But there are some significant challenges in this field~\cite{whyrust,Shapiro:2006:PLC:1215995.1216004}
\end{itemize}
\end{frame}

\begin{frame}
\huge What is Mnemosyne? \normalsize \\
A functional systems programming language with \alert<1->{compile-time memory management}.
\begin{itemize}
    \item<1-> We'll discuss this in a moment
    \item<1-> First, some background
\end{itemize}
\end{frame}


\begin{frame}
    \huge References \normalsize \\

    \printbibliography
\end{frame}

\end{document}
