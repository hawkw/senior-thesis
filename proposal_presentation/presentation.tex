\documentclass{beamer}
% \usepackage{listings}
\usepackage{color}

%% -- Referencing -------------------------------------------------------
\usepackage{hyperref}
\usepackage{cleveref}
\usepackage[ backend=bibtex
           , style = numeric
           , sortcites = true
           , url = false
        %    , doi = false
           , sorting = nty
           , backref
           , natbib
           , hyperref
           ]{biblatex}
\bibliography{../bibliography/thesis.bib}
\renewcommand*{\bibfont}{\footnotesize}

\usecolortheme[dark]{solarized}

\title{\huge Mnemosyne}
\subtitle{A Functional Systems~Programming Language}
\author[Hawk Weisman]{Hawk Weisman}
\institute[Allegheny College]{Department of Computer Science \\ Allegheny College}
\date{\today}
\begin{document}
\maketitle

\begin{frame}
\huge What is Mnemosyne? \normalsize \\
\A functional systems programming language with compile-time memory management.
\begin{itemize}
    \item But what does that mean?
\end{itemize}
\end{frame}

\begin{frame}
\huge What is Mnemosyne? \normalsize \\
A \alert<1-3>{functional} systems programming language with compile-time memory management.
\begin{itemize}
\item<2-3>
    \textbf{Functional programming} models computation as the evaluation of functions~\cite{Wise:2003:FP:1074100.1074416}
    \item<2-3> Focus on \textit{purity} and \textit{immutability}
    \item<3> Advantages: \begin{itemize}
        \item<3> Expressive
        \item<3> Easy to tests
        \item<3> Safe
        \item<3> Paralellizable
    \end{itemize}
\end{itemize}
\end{frame}

\begin{frame}
\huge What is Mnemosyne? \normalsize \\
A functional \alert<1->{systems programming} language with compile-time memory management.
\begin{itemize}
\item<1->
    \textbf{Systems programming} is the implementation of software that provide services to other software~\cite{Narten:2003:SP:1074100.1074850,Shapiro:2006:PLC:1215995.1216004}.
    \begin{itemize}
    \item<2>{Operating systems}
    \item<2>{Device drivers}
    \item<2>{Language runtimes}
    \item<2>...\end{itemize}
\end{itemize}
\end{frame}

\begin{frame}
    \huge References \normalsize \\

    \begingroup
    {\small
    \setlength{\emergencystretch}{1em} % this fixes overfull hboxes in citations
                                       % (according to the interwebs)
    \printbibliography}
    \endgroup
\end{frame}

\end{document}
